% Prompt ChatGPT pour la création des block forest :

% Tu es un assistant LaTeX.
% Transforme le listing de répertoire ci-dessous en un **bloc LaTeX `forest`** prêt à coller, en utilisant UNIQUEMENT les styles/macros DÉJÀ définis dans mon préambule :

% - Environnement : `\begin{forest} base layout,compact ... \end{forest}`
% - Styles de nœud : `folder` (dossiers), `file` (fichiers)
% - Macros : `\froot{...}`, `\fdir{...}`, `\ffile{...}`, `\fcode{...}`, `\fimage{...}`
% (Elles gèrent les icônes ON/OFF et la marge `\iconpad`. **Ne les redéfinis pas**.)

% RÈGLES DE CONVERSION
% 1) **Entrée** : bloc brut (ex: sortie `tree`). Le premier niveau est la **racine**.
% 2) **Type de nœud** :
% - Dossier → style `folder` + macro `\fdir{...}` (ou `\froot{...}` pour la racine).
% - Fichier → style `file` et macro selon l’extension :
% * Code : `*.php, *.json, *.js, *.ts, *.css, *.html, *.md, *.xml, *.yml, *.yaml, *.ini` → `\fcode{...}`
% * Image : `*.png, *.jpg, *.jpeg, *.gif, *.svg, *.webp` → `\fimage{...}`
% * Sinon → `\ffile{...}`
% 3) **Échapper** les caractères LaTeX spéciaux dans les libellés :
% `_ -> \_`, `% -> \%`, `# -> \#`, `& -> \&`, `{ -> \{`, `} -> \}`, `~ -> \textasciitilde{}`, `^ -> \textasciicircum{}`, `\ -> \textbackslash{}`
% 4) **Structure** : respecte l’imbrication. Un dossier devient `[ ... , folder  ... ]` contenant ses enfants.
% 5) **Racine** : utiliser `[ \froot{<NomRacine>}, folder  ... ]`.
% 6) (Optionnel) Trier : dossiers avant fichiers, ordre alphabétique.
% 7) **Sortie attendue** : **uniquement** le bloc `forest`, sans texte additionnel.

% EXEMPLE DE SORTIE ATTENDUE
% \begin{forest} base layout,compact
% 	[\froot{projet}, folder
% 	[\fdir{public}, folder
% 	[\fcode{index.php}, file]
% 	[\ffile{.htaccess}, file]
% 	]
% 	[\fdir{src}, folder
% 	[\fdir{Controllers}, folder]
% 	[\fdir{Models}, folder]
% 	[\fdir{Views}, folder]
% 	[\fdir{Routes}, folder]
% 	]
% 	[\fcode{composer.json}, file]
% 	]
% \end{forest}

% LISTING À CONVERTIR (ne modifie pas ces deux lignes) :
% BEGIN LISTING
% (colle ici la sortie `tree` ou autre)
% END LISTING

% --- packages ---
\usepackage{forest}
\usepackage{fontawesome5}      % icônes FA5

% --- encadré esthétique (optionnel) ---
% \newtcolorbox{treebox}{
% 	colback=white, colframe=black!10, boxrule=0.4pt,
% 	borderline west={2pt}{0pt}{blue!35},
% 	arc=2mm, left=8pt, right=8pt, top=8pt, bottom=8pt
% }

% --- palette ---
\definecolor{pill}{HTML}{EEF2FF}
\definecolor{pillborder}{HTML}{C7D2FE}
\definecolor{filepill}{HTML}{F3F4F6}
\definecolor{fileborder}{HTML}{E5E7EB}
\definecolor{ink}{HTML}{0F172A}

% --- interrupteur + marge dépendante ---
\newif\ificons
\newdimen\iconpad
% réglage global simple : \SetIcons{true} ou \SetIcons{false}
\newcommand{\SetIcons}[1]{%
	\csname icons#1\endcsname
	\ificons \iconpad=1.1em\relax \else \iconpad=0.3em\relax \fi
}
% init : sans icônes
\SetIcons{false}

% --- base commune forest ---
\forestset{
	base layout/.style={
		for tree={
			font=\sffamily\small\color{ink},
			grow'=0,
			parent anchor=east, child anchor=west,
			edge path'={(!u.parent anchor) -- +(7pt,0) |- (.child anchor)},
			edge={draw=black!25, line width=.5pt},
			l sep=10pt, s sep=8pt, anchor=west, calign=first,
			text height=1.2ex, text depth=.2ex,
			inner xsep=\iconpad,  % marge commune (dépend de \SetIcons)
		},
		pill/.style={draw, rounded corners=4pt, minimum height=1.4em},
		folder/.style={pill, fill=pill,     draw=pillborder},
		file/.style  ={pill, fill=filepill, draw=fileborder},
	},
	compact/.style={for tree/.append style={l sep=8pt, s sep=5pt}},
}

% --- icônes : fallback si rien n’est indiqué ---
\newcommand{\DefaultIcon}{file} % icône par défaut

% case d’icône : réserve la place ; affiche l’icône si \iconstrue
\newcommand{\IconBox}[1]{%
	\makebox[\iconpad][c]{\ificons \faIcon{#1}\fi}%
}
% version “peut-être vide” : si argument vide => DefaultIcon
\newcommand{\IconBoxMaybe}[1]{%
	\if\relax\detokenize{#1}\relax
	\IconBox{\DefaultIcon}%
	\else
	\IconBox{#1}%
	\fi
}

% --- macros de label (renommées) ---
\newcommand{\froot}[1]{\IconBox{folder-open}#1}
\newcommand{\fdir}[1]{\IconBox{folder}#1}
\newcommand{\ffile}[1]{\IconBox{file}#1}
\newcommand{\fcode}[1]{\IconBox{file-code}#1}
\newcommand{\fimage}[1]{\IconBox{image}#1} % ou "file-image"

% (optionnel) générique : \withicon[<nom-ou-vide>]{Texte}
\newcommand{\withicon}[2][]{\IconBoxMaybe{#1}#2}